% compile with XeTeX
\documentclass[11pt]{article}

\usepackage{NotesTeX}


\usepackage{geometry}
\usepackage{amsmath}
\usepackage{amsfonts}
%\usepackage[regular]{newcomputermodern}
\usepackage{unicode-math}
%\setmathfont[range=\mathbb]{TeX Gyre Termes Math}
%\usepackage{mathtools}
%\usepackage[T1]{fontenc}
\usepackage{cancel}
\usepackage{graphicx}
\usepackage{xcolor}

%\usepackage{fancyhdr}

\usepackage{slashed} % for Dirac operators

\usepackage{hyperref}
\hypersetup{colorlinks=true, linkcolor=blue, citecolor=blue, urlcolor=blue}

%\geometry{margin=2.5cm, top=2.5cm}
\author{Aayush Arya}
\title{Practical Introduction to Experimental High Energy Physics}
\affiliation{Johannes Gutenberg-Universitat Mainz}
\emailAdd{aarya@students.uni-mainz.de}
%\date{}

\newcommand{\lag}{\mathcal{L}}
\newcommand{\ham}{\mathcal{H}}

\begin{document}
	
	\maketitle
	
	\pagestyle{fancynotes}
	
	%\vspace{-8em}
	\part*{Foreword}
	These notes were crafted by me as a beginning master's student at Mainz. My professors Lucia Masetti and Volker Buescher at JGU taught me the content while field-testing a specialized course ``Practical Introduction to Experimental High Energy Physics".
	
	\newpage
	
	\part{Phenomenology of Collider Physics}
	
	\newpage
	\part{Detectors}
	
	\newpage
	\part{Statistical Methods}
	
	\newpage
	\part{Extracting physics from LHC data}
	\newpage
	\section{Data acquisition and preparation}
	
	\newpage
	\section{Particle reconstruction}
	
	\newpage
	\section{Particle identification}
	
	\newpage
	\section{Detector calibration}
	
	\newpage
	\section{Cross section measurements}

\subsection{Inclusive cross section}

\subsubsection{lol?}

\subsection{Differential and Fiducial cross-section}
	
	\newpage
	\part{Searches for physics beyond the Standard Model}
	


\end{document}