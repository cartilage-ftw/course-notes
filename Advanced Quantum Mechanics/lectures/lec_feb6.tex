% compile with XeTeX
\documentclass[11pt]{article}

\usepackage{physics} 

\usepackage{geometry}
\usepackage{amsmath}
\usepackage{amsfonts}
%\usepackage[regular]{newcomputermodern}
\usepackage{unicode-math}
%\setmathfont[range=\mathbb]{TeX Gyre Termes Math}
%\usepackage{mathtools}
%\usepackage[T1]{fontenc}
\usepackage{cancel}
\usepackage{graphicx}
\usepackage{xcolor}

\usepackage{fancyhdr}

\usepackage{slashed} % for Dirac operators

\usepackage{hyperref}
\hypersetup{colorlinks=true, linkcolor=blue, citecolor=blue, urlcolor=blue}

\geometry{margin=2.5cm, top=2.5cm}
\author{}
\title{}
\date{}

\newcommand{\lag}{\mathcal{L}}
\newcommand{\ham}{\mathcal{H}}

\begin{document}
	
	\pagestyle{fancy}
	\maketitle
	
	\vspace{-8em}
	
	
	%\hrule
	\noindent
	%\hspace{-1.2em}
	\fbox{
		\parbox{\textwidth}{
			\begin{center}
				\vspace{-8pt}
				\footnotesize{Advanced Quantum Mechanics}\hfill \footnotesize{WiSe 2022/23}\\
				%\vspace{4pt}
				\centering \normalsize Aayush Arya \\
				\centering \small Feburary 6, 2023\\
				\footnotesize{Theoretical Physics 5} \hfill \footnotesize{Instructor: Harvey Meyer}
			\end{center}
			\vspace{-8pt}
		}
	}
	%\hrule 
	
	\vspace{11pt}
	
	
	Continuing the math from last lecture
	
	\[ H = \hbar c^2 \int \frac{d^3 k}{(2\pi)^3} \frac{k^0}{2\omega_k} \sum_{r_s} (a^r(\vec{k})^\dagger u^r(\vec{k})^\dagger + b^r(-\vec{k})v^r(-\vec{k})^\dagger )(a^s(\vec{k})u^s(\vec{k}) - b^s(-\vec{k})^\dagger v^s(-\vec{k}))  \]
	
	
	\[ \text{and, since}\ \quad \begin{cases}
		u^s(\vec{k})^\dagger u^s (\vec{k}) \\
		v^r(\vec{k}) v^s(\vec{k}) 
	\end{cases} =\ 2k^0 \delta^{rs}
	\]
		\[u^r(\vec{k})^\dagger v^s(-\vec{k}) = 0\]
	
	the double sum collapses
	
	\[ = \hbar c^2 \int \frac{d^3k}{(2\pi)^3\ \cancel{2k^0}}\frac{k^0 \cancel{(2k^0)}}{c}
.		\sum_{s=1,2} (a^{s\dagger}_k)a^s_k - \underline{b^s_{-k} b_{-k}^{s\dagger}} \]
	
	we rewrite the underlined part as $-b^r(-\vec{k})^\dagger b^s(-\vec{k}) + (2\pi)^3\delta^{}$.\\
	
	It's immediately worth remembering what the constant term in that does \textemdash it adds a zero-point vacuum energy, which makes the Hamiltonian diverge. If we are only interested in ``differences" in energy (which are what we can observe) then such a pathology can be removed by using a constant energy density term that exactly cancels this. In essence,
	
	\begin{center}
		\boxed{
		H = \int \frac{d^3k}{(2\pi)^3} \hbar \omega_k \sum_{s=1,2}(a_k^{s\dagger}a^s_k + b_k^{s\dagger} b^s_k)	+ \text{(constant)}
	}
	\end{center}
	
	The spatial momentum operator \[ \vec{P} = \frac{1}{\hbar c}\int d^3\vec{x} \psi^\dagger(x) (-i\hbar -\vec{\nabla}) \psi(x)\]
	
	Notice the similarities (and differences) of this with the Hamiltonian.
	We can pull out an $\hbar$, multiply with $\gamma^0$ (to turn $\psi^\dagger \to \bar{\psi})$
	
	\[ \vec{P} = \frac{1}{c}\int d^3\vec{x} \bar{\psi}(x)(-i\gamma^0 \vec{\nabla}) \psi(x) \]
	
	We can then insert the expansion for $\bar{\psi}$ and $\psi$
	
	\[ = \hbar c^2 \frac{d^3k}{(2\pi)^3} \frac{\vec{k}}{2\omega_k} \sum_{r,s} (a^r(\vec{k})^\dagger u^r(\vec{k})^\dagger + b^r(-\vec{k}) v^r(-\vec{k})^\dagger)(a^s(\vec{k}u^s(\vec{k}) + b^s(-\vec{k})^\dagger v^s(-\vec{k})))\]
	
	$$:$$\\
	
	and an analogous treatment to get
	
	\begin{center}
		\boxed{\vec{P} = \int \frac{d^3 \vec{k}}{(2\pi)^3}\hbar \vec{k} \sum_{s=1,2} (a^s(\vec{k})^\dagger a^s(\vec{k}) + b^s(\vec{k})^\dagger b^s(\vec{k}))
		}
	\end{center}
which is the total momentum carried by electrons and positrons.
	
	We also have a third conserved quantity in the Dirac theory (a current), just like in the complex scalar field theory
	
	\[ j_\mu = \frac{1}{i} (\phi \partial_\mu \phi^* - \phi^* \partial_\mu \phi)\]
	obeys $\partial_\mu j^\mu = 0$
	
	\[ \Rightarrow\ Q = \int d^3\vec{x} j^0\]
	which is a time-independent (conserved) quantity.\\
	
	
	In scalar field theory, when we had a Lagrangian
	
	\[ \lag = \partial_\mu \phi \partial^\mu \phi^* - \kappa^2 \phi^* \phi\]
	which was invariant under a phase shift by $e^{i\alpha}$
	
	\begin{align*}
		\phi(x) &\to e^{i\alpha} \phi(x)\\
		\phi^*(x) &\to \phi^*(x) e^{-i\alpha}
	\end{align*}

	
	similarly, here our $\lag = \bar{\psi}(i\gamma^\mu \partial_\mu - \kappa)\psi$ is also under the same type of rotation, i.e. it preserves a $U(1)$ symmetry \footnote{This is not a conformal symmetry, by the way. A conformal transformation also involves $x$ (the argument of the field)}
	
	The associated conserved current satisfying $\partial_\mu = 0$ is $\bar{\psi}\gamma^\mu \psi$
	
	and it's trivial to see that  \begin{align*}
		\partial_\mu (\bar{\psi}\gamma^\mu \psi) & = \bar{\psi}\gamma^\mu \partial_\mu + \bar{\psi}\gamma^\mu\partial_\mu \psi\\
		& =\bar{\psi}(-\kappa)\psi + \bar{\psi}\kappa \psi\\
		& =0
	\end{align*}

	and the conserved charge then
	
	\[ Q = \int d^3 \vec{x} \bar{\psi}\gamma^0\psi\]
	
			\[ = \hbar c^2 \int \frac{d^3 K}{(2\pi)^3}\frac{2k^0}{2k^0 c} \sum_{s=1,2}( ... )\]
	
	
	and we apparently end up getting
	
	\[ Q = \hbar c \int \frac{d^3 k}{(2\pi)^3} \sum_{s=1,2}(a^{s\dagger}_k a^s_k - b^{s\dagger}_k b^s_k)\]
	
	Note interestingly how we have terms of opposite signs here.\\
	
	This is saying that the electric charge of the electrons and positrons act oppositely to each other.\\
	
	The number of electrons $a^{s}(\vec{k})^\dagger a^s(\vec{k})$ alone is not conserved (one can have pair creation and annihilation processes, it's the total charge that would still be conserved).\\
	
	Electric charge operator: $Q_{\rm{elec}} = eQ$\\
	
	Interestingly, there's another transformation of the Dirac field
	
	\[ \gamma_5 = i\gamma^0 \gamma^1 \gamma^2\gamma^3 = \begin{pmatrix}
		-\mathbb{1} & 0 \\ 0 & \mathbb{1}
	\end{pmatrix}\]
	
	\[ \psi_L = \frac{1-\gamma_5}{2}\psi, \quad \psi_R = \frac{1+\gamma_5}{2}\psi \]
	  
	  \[ \psi^* \gamma^0 \psi = \psi_l^* \psi_R + \psi^*_R\psi_L\]
	  
	  
	  An important property of the $\gamma_5$ matrix is that it anti-commutes with every other one of the $\gamma^\mu$ matrices, i.e. 
	  
	  \[ \{\gamma_5, \gamma^\mu\} = 0\]
	  
	  Let's try another transformation of the field and see if that leads to a symmetry
	  
	  \begin{align*}
	  	\psi \to & e^{i\alpha \gamma_5} \psi\\
	  	\bar{\psi} \to & \bar{\psi} e^{i\alpha\gamma_5}
	  \end{align*}
  
  		In $\bar{\psi} \gamma^\mu \psi \to \bar{\psi} e^{i\alpha\gamma_5}\gamma^\mu e^{i\alpha\gamma_5} \psi = \bar{\psi} \gamma^\mu\psi $
  		
  		The odd terms in the expansion of $e^{i\alpha \gamma_5}$ pick up a minus sign. So, $e^{i\alpha \gamma_5}\gamma^\mu = \gamma^\mu e^{-i\alpha\gamma_5}$.
	  
	  Although the mass term $\bar{\psi}\psi \to \bar{\psi} e^{2i\alpha \gamma_5} \psi$ is not invariant.
	  The kinetic term $\bar{\psi}\gamma^\mu \partial_\mu\psi \to \bar{\psi} \gamma^\mu \partial_\mu \psi$ is invariant.
	  
	  This is still a symmetry for massless particles though.\\
	  
	  
	  Consider the current $\bar{\psi}\gamma_5 \gamma^\mu \psi$
	  
	  \[ \partial_\mu (\bar{\psi} \partial_\mu \gamma_5 \gamma^\mu \psi + \bar{\psi} \gamma_5 \gamma^\mu\partial_\mu \psi)  = \underbrace{ - \bar{\psi}(\partial_\mu \gamma^\mu) \gamma_5 \psi}_{-\kappa\bar{\psi \gamma_5 \psi}}\ +\ \bar{\psi}\gamma_5 \kappa \psi \]
			
			\[ = 2 \kappa\bar{\psi}\gamma_5\psi\]

	which vanishes only for $\kappa=0$ (massless particles).

\end{document}