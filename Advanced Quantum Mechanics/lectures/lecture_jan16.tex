\documentclass[11pt]{article}

\usepackage{geometry}
\usepackage{amsmath}
\usepackage{amsfonts}
\usepackage{unicode-math}
%\usepackage{mathtools}
%\usepackage[T1]{fontenc}
\usepackage{cancel}
\usepackage{physics}
\usepackage{graphicx}
\usepackage[dvipsnames]{xcolor}
\usepackage{hyperref}
\hypersetup{colorlinks=True, linkcolor=blue, citecolor=blue, urlcolor=blue}
\urlstyle{same}

\geometry{margin=2.5cm, top=2.5cm}
\author{}% Aayush Arya
\title{}
\date{} % Jan 15

\newcommand{\lag}{\mathcal{L}}
\newcommand{\ham}{\mathcal{H}}

\usepackage{fancyhdr}
%\usepackage{titling}
%\setlength{\droptitle}{-10em}
\begin{document}
	\maketitle
	\vspace{-9em}

	%\hrule
	\noindent
	%\hspace{-1.2em}
	\fbox{
		\parbox{\textwidth}{
			\begin{center}
				\vspace{-8pt}
				\footnotesize{Advanced Quantum Mechanics}\hfill \footnotesize{WiSe 2022/23}\\
				%\vspace{4pt}
				\centering \normalsize Aayush Arya \\
				\centering \small January 16, 2023\\
				\footnotesize{Theoretical Physics 5} \hfill \footnotesize{Instructor: Harvey Meyer}
			\end{center}
			\vspace{-8pt}
		}
	}
	%\hrule 
	
	\vspace{11pt}
	
	\section*{The Lorentz Group}
	
	\[ x'^{\mu} = \Lambda^\mu_\nu x^\nu \quad \text{(homogenous Lorentz transf.)}\]

with \[ \Lambda^T g \Lambda = g, \quad g = \text{diag}(1, -1, -1, -1)	\]
	
	The proper orthochronous Lorentz group $\lag^+$ is a continuous, connected, 6-parameter group  (proper: $\det \Lambda = + 1$, orthochronous: $\Lambda^0_0 \geq 1 $)\\
	
	3 rotation directions, 3 boost directions.\\ \vspace{-11pt}
	{\hfill \color{VioletRed} \rule{0.5\textwidth}{0.4pt} \hfill }\\
	
	\noindent {\textcolor{VioletRed}{\large \bf My understanding:}}
	{\color{darkgray}
		We need transformations that we can apply on states $\psi$ that live in a Hilbert space, rather than a general Euclidean space. We will also want our transformations to satisfy certain properties, e.g. 
		\vspace{-4pt}
		\begin{itemize}
			\item it should not affect the probability density associated with a state.\\
			 Say $\hat{T}(a) \ket{\psi(x)} = \ket{\psi(x+a)}$, then we would like that  %\vspace{-4pt}
			 $$\bra{\psi(x)}\ket{\psi(x)} = \bra{\psi(x+a)}\ket{\psi(x+a)} = \bra{\psi(x)}\hat{T}^\dagger(a)\hat{T}(a)\ket{\psi(x)}$$
			 %\vspace{-6pt}
			 This requires $\hat{T}^\dagger\hat{T} = 1$, i.e. $\hat{T}$ should be unitary.
			 \vspace{-6pt}
			 \item A composition rule such as $\hat{T}(a)\hat{T}(b) = \hat{T}(a+b)$ to be obeyed
			 \vspace{-6pt}
			 \item $\hat{T}(0) = \mathbb{1}$, such that no transfomation leaves things unaffected.
		\end{itemize}
	\vspace{-4pt}
	This means that our transformations should form a group $G$. Whenever we have a ``continuous" group of the sorts involved here, it will follow the structure of a \textit{Lie algebra}. \vspace{4pt} \\
	It should be possible to write every element of a \textit{Lie group} $g \in G$ as $g(\vec{\alpha}) = 1 + i\alpha^i T^i + O(\alpha^2)$, where the \textit{generators} of the group encode the algebra in their commutator
	 $[T^i, T^j] = i f^{ijk} T^k$. The \textit{structure constant} $f^{ijk}$ in our cases will be $\epsilon^{ijk}$.\vspace{4pt} \\ Our job is then to find a \textit{representation} for our group appropriate for the objects we are dealing with (scalar, vector, or spinor fields).
	}

	{\hfill \color{VioletRed} \rule{0.5\textwidth}{0.4pt} \hfill }\\
	\vspace{-24pt}
	\subsection*{Infinitesimal transformations}
	\begin{align*}
		\Lambda_R(\alpha, \hat{e_z}) & = \begin{pmatrix}
			1 & 0 & 0 & 0\\
			0 & 1 & -\alpha & 0\\
			0 & \alpha & 1 & 0\\
			0 & 0 & 0 & 1\\
		\end{pmatrix}\\	
	& = \mathbb{1} + \alpha \begin{pmatrix}
		0 & 0 &0& 0\\
		0 & 0 & -1 & 0\\
		0 & 1 & 0 & 0 \\
		0 & 0& 0& 0\\
	\end{pmatrix}
	\end{align*}
	 

where the matrix on the right is the generator of rotations around the z-axis.

\begin{align*}
	\Lambda_B (\eta, \hat{e_1}) & \simeq \begin{pmatrix}
		1 & \eta & 0 & 0\\
		\eta & 1 & 0 & 0\\
		0 & 0 & 1 & 0\\
		0 & 0&  0 & 1\\
	\end{pmatrix}\\
		& = \mathbb{1}_{4\times 4} + \eta \begin{pmatrix}
		0 & 1 & 0 & 0\\
		1 & 0 & 0 & 0\\
		0 & 0 & 0 & 0\\
		0 & 0 & 0 & 0\\
	\end{pmatrix}
\end{align*}


and here it serves as a generator  of a boost along the x-axis.

In general \[ \Lambda^\alpha_\beta = \delta^\alpha_\beta - \frac{i}{2} \omega_{\mu\nu} (J^{\mu\nu})^\alpha_\beta\]

where the $\omega_{\mu\nu} = - \omega_{\nu\mu}$ has the parameters of the infintesimal Lorentz transformation (6 indep. parameters)

and $(J^{\mu\nu})^\alpha_\beta$ is the corresponding generator to $\omega_{\mu\nu}$. \[ \exists\ \text{6  indep. generators} (J^{\nu\mu} = - J^{\mu\nu})\]


so we could have also written \[ \frac{1}{2} \sum_{\mu,\nu} \equiv \sum_{\mu < \nu} S_{\{\mu, \nu\}}\]	


so what is this $(J^{\mu\nu})^\alpha_\beta$?

\[ (J^{\mu\nu})^\alpha_\beta = i(g^{\mu\alpha} \delta ^\nu_\beta - g^{\nu\alpha} \delta^\mu_\beta) \]

\[ J^{12} = i \begin{pmatrix}
	0 & 0 & 0 & 0\\
	0 & & &0\\
	0 & & & 0\\
	0 & 0 & 0 & 0
\end{pmatrix}\]

and the $2\times2$ block is what we have to figure out

\[ (J^{12})^\alpha_\beta =
 i(g^{1\alpha} \delta^{2}_{\beta} - g^{2\alpha} \delta^1_\beta )\]
 
 
 the box should come out as \[\begin{matrix}
 	0 & 1 \\
 	-1 & 0\\
 \end{matrix}\]
 
 
 In $ J^{01}$ only the $(0, 0) \rightarrow (1, 1)$ corner of the matrix gets to be non-zero, and the rest are zero
 
 \[ J^{01} = i \begin{pmatrix}
 	0 & +1 & 0 & 0\\
 	+1 & 0 & 0 & 0\\
 	0 & 0 & 0& 0\\
 	0 & 0 & 0& 0\\
 \end{pmatrix}\]
 
 
 what does the $\omega_{12}$ correspond to? It corresponds to \[\omega_{12} = \alpha \quad \text{for the rotation around the z-axis}\]
 
 \vspace{-16pt}
 \[ \omega_{01} = \eta \quad \text{for the boost along x-axis}\]
 
 In special relativity when we have two rotations $\rightarrow$ equiv. to just a rotation.
 
 The composition laws of Lorentz transformations are encoded in the commutation relations of the generators $J^{\mu\nu}$
 
 
 \[ [ J^{\mu\nu}, J^{\rho\sigma}] = i(g^{\nu\rho}J^{\mu\sigma} - g^{\mu\rho} J^{\nu\sigma} - g^{\nu\sigma} J^{\mu\rho} + g^{\mu\sigma} J^{\nu\rho})\]
 
 where the $J$'s are $4\times 4$ matrices and $g$'s are the coefficients.
 
 \[ \Lambda_R(-\alpha, \hat{e_z}) \Lambda_B(\eta, \hat{e_1}) \Lambda_R(\alpha, \hat{e_z}) \]
 
 these can be written as \[ \left( 1 + \frac{i\omega_{12}}{2}J^{12} \right) \left( 1 - \frac{i}{2} \omega_{01}J^{01}\right)  \left(1 - \frac{i}{2} \omega_{12} J^{12}\right) \]
 
 which is also equal to \[ \mathbb{1} - \frac{i}{2}\omega_{01}J^{01} + \frac{1}{4} \omega_{12}\omega_{01} \underbrace{(J^{12} J^{01} - J^{01}J^{12})}_{\text{commutator}\ [J^{12}, J^{01}]} + \mathcal{O}(\omega_{12}^2)\]
 
 where the underbraced part is exactly the commutator $[J^{12}, J^{01}]$
 
 all these group properties are encoded in \\
 
 $\rightarrow$ A boost and a rotation commute only if the axis of the rotation is also the direction of the boost. The boost doesn't affect the transverse plane.\\

	This sort of transformation law for the field $\psi$ is called the \textit{representation} of the Lorentz group.
	
	These are also relevant in condensed matter physics, when there are symmetries in a crystal, for instance. When there's a finite number of symmetry tranformations you can do, you come up with a \textit{representation} for a \textit{discrete group}.
	
	\section*{Representation of the Lorentz group}
	
	\[ \psi_\alpha '(x') = \mathcal{M}(\Lambda)_{\alpha\beta}\ \psi'_\beta (x')\]
  This expresses that the field $\psi$ is a representation of the Lorentz group\\
  
	suppose $\Lambda = \Lambda_1 \Lambda_2$ (composition of 2 Lorentz tranformations). If you do transformation $\Lambda_2$
	
	\[ \psi(x) \xrightarrow{\Lambda_2} \mathcal{M} (\Lambda_2) \psi(x) \xrightarrow{\Lambda_1} M(\Lambda_1) (M(\Lambda_2)\psi(x))\]
	
	that's the group property (the result of the composition is again a Lorentz transformation).
	
	This is equiv. to 
	\[ \psi(x) \xrightarrow{\Lambda} M(\Lambda) \psi(x)\]
	
	The consistency requires that \[ M(\Lambda_1\Lambda_2) \equiv M(\Lambda_1) M(\Lambda_2)\]


	We have seen two examples of $\mathcal{M}(\Lambda)$ 
	
	\begin{itemize}
			\item if $\mathcal{M}(\Lambda) = 1$ (works for scalars) then that's called the \text{trivial representation}\footnote{A trivial representation exists for every group} of the Lorentz group
			
			\item $\mathcal{M}(\Lambda) = \Lambda$: the fundamental representation of the Lorentz group
			
	\end{itemize}

	the dimension of the representation is the (minimal) number of field components it has, in this case, 4-dimensional.\\
	
	\boxed{\rightarrow \text{These will not do to describe the electron}} \footnote{Since rotations about axis $i$ are generated by the angular momentum operator $J^i$, a representation $D(\theta^i) = e^{-i\hat{J}^i\theta^i}$ should recover $J^i = -\frac{1}{i} \frac{\partial}{\partial \theta^i}D(\theta^i)$. For an electron this needs to be something that recovers $J^i = \pm 1/2$ in units of $\hbar$. A good example is $\frac{1}{2}\sigma^i$ where $\sigma^i$ are the Pauli matrices.}\\
	
	For an infinitesimal transformation, we can expect that \[ \mathcal{M}(\Lambda) = \mathbb{1} - \frac{i}{2} \omega_{\mu\nu} \bar{\mathcal{J}}^{\mu\nu} \] 

	$\rightarrow$ the $\bar{\mathcal{J}}^{\mu\nu}$ have the same commutation relations as the $J^{\mu\nu}$'s\\

	Now we come to one thing that we know about the electron (spin-1/2). Spin is about angular momentum, which has to do with spatial rotations. So we know something about how electron transforms under rotations just from that it has spin-1/2.\\

	For rotations, there's a connection between $J^{ij}$ and the angular momentum
	
	\[ J^{ij} = \frac{1}{\hbar}\epsilon^{ijk} S^{k}\]

for angular momenta, commutation relations: \[ [S^i, S^j] = i \hbar \epsilon^{ijk} S^k\]

	In spin-1/2 we use Pauli matrices. \[ S^k = \hbar \frac{\sigma^k}{2}\]
	
	where the $\sigma^k$ are Pauli matrices
	
	\[ \sigma^1 = \begin{pmatrix}
		0 & 1 \\ 1 & 0
	\end{pmatrix} \quad \sigma^2 = \begin{pmatrix}
	0 & -i \\ i & 0
\end{pmatrix} \quad \sigma^3 = \begin{pmatrix}
1 & 0 \\ 0 & -1
\end{pmatrix}\]

For the rotations, this already tells us what the answer is of what the $\bar{\mathcal{J}}^{ij}$ should be. When it came to including boosts, it was Dirac who had the brilliant idea

\section*{Dirac construction of a representation for the electron}

Introduce 4 matrices $\gamma^\mu,\ \mu \in \{0, 1, 2,\ ..\}$ satisfying that the anti-commutator

\[ \{ \gamma^\mu, \gamma^\nu \} = \gamma^\mu\gamma^\nu + \gamma^\nu\gamma^\mu =  (2g^{\mu\nu})\mathbb{1}_{n\times n} \]

This is essentially saying the square of the matrix should be the unit matrix.

If we can find such a matrix, then the claim is \[ \bar{\mathcal{J}}^{\mu\nu} := \frac{i}{4} [\gamma^\mu, \gamma^\nu]\]

in other words, they can serve as the generators of the representations of the Lorentz group.


And it turns out that minimal size of the matrices is 4. This is a coincidence, because in d-dimensions, it is $2^{d/2}$ (that is to say, in 8-dimensions, it would be $2^4 = 16$).


\[ J^{ij} = \epsilon^{ijk} \begin{pmatrix}
	\sigma^k & 0 \\ 0 & \underbrace{\sigma^k}
\end{pmatrix} \]
where the underbraced component corresponds to the \textit{anti-particle}
\end{document}

