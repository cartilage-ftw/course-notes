% compile with XeTeX
\documentclass[11pt]{article}

\usepackage{geometry}
\usepackage{amsmath}
\usepackage{amsfonts}
%\usepackage[regular]{newcomputermodern}
\usepackage{unicode-math}
%\setmathfont[range=\mathbb]{TeX Gyre Termes Math}
%\usepackage{mathtools}
%\usepackage[T1]{fontenc}
\usepackage{cancel}
\usepackage{graphicx}
\usepackage{xcolor}

\geometry{margin=2.5cm, top=1cm}
\author{Aayush Arya}
\title{}
\date{January 26, 2023}

\newcommand{\lag}{\mathcal{L}}
\newcommand{\ham}{\mathcal{H}}

\begin{document}
	\maketitle
	
	\hrule
	\begin{center}
		Lecture Notes\\
		Advanced Quantum Mechanics
	\end{center}
	\hrule 
	
	\vspace{11pt}
	
	\subsection*{The Dirac equation}
	\[ (i\gamma^\mu \partial_\mu - \kappa)\psi(x) = 0\]
		
		\begin{itemize}
			\item derives from the Lagrangian density $\lag = \bar{\psi} (i\gamma^\mu \partial_\mu)\psi$
			\item implies the Klein-Gordon equation, $(\Box + \kappa^2) \psi = 0$
		\end{itemize}
		
		$k^\mu$ and $p^\mu$ are two four vectors
			\begin{align*}
				p_\mu \gamma^\mu k_\nu \gamma^\nu &= \frac{1}{2}(p_\mu k_\nu \gamma^\mu \gamma^\nu + p_\nu k_\mu \gamma^\nu \gamma^\mu)\\
				    & \text{we can replace}\ \gamma^\mu \gamma^\nu = -\gamma^\nu \gamma^\mu + 2g^{\mu\nu} \mathbb{1}_{4\times 4}
			\end{align*}
		which then takes the neat expression
		%\vspace{-11pt}
		\begin{center}
			\boxed{(p\cdot \gamma) (k\cdot \gamma) = (p\cdot k) \mathbb{1}_{4\times 4}}
		\end{center}
	\subsection*{Hamiltonian for the Dirac field: $\bar{\psi}\equiv \psi^\dagger\gamma^0$}
	
	\[ \frac{\partial \lag}{\partial \psi_\alpha} = \frac{1}{c} \psi^\dagger_\alpha = \Pi_\alpha, \quad \underbrace{(\ ... \ )}_{\bar{\psi}} \begin{pmatrix} .. & . & . & . \\ : &  & & : \\
			. &  & & .\\ 
			. & & & .\end{pmatrix}
		\underbrace{\begin{pmatrix}
				: \\ .\\. \\.
		\end{pmatrix}}_{\psi} \]
	
	We obtain the Hamiltonian by the Legendre tranformation $\ham = \sum_\alpha \left( \Pi^0_\alpha \dot{\psi_\alpha} \right) - \lag$. In our case, since $\Pi^0$ for $\bar{\psi}$ is 0, the summation only the one corresponding to $\psi$ appears in the expression.
	\begin{align*}
		H &= \int d^3 \vec{x} \left( \frac{i}{c}\psi^\dagger_\alpha \dot{\psi}_\alpha - \lag \right)\\
			& = \int d^3 \vec{x} \left( \cancel{ \frac{i}{c}\psi^\dagger \dot{\psi} }- (  \cancel{ \bar{\psi}i\gamma^0 \partial_0\psi } + \bar{\psi} i \gamma^k \partial_k \psi - \kappa \bar{\psi} \psi) \right)\\
		H &= \int d^3 \vec{x}\ \bar{\psi} (-i\vec{\gamma}\cdot \vec{\nabla} + \kappa)\psi, \quad \text{with}\ \vec{\gamma} = (\gamma^1, \gamma^2, \gamma^3)^T = -\vec{\gamma}^\dagger
	\end{align*}
	from the Dirac equation itself $i\gamma^0\partial_0 = -i\vec{\gamma}\cdot\vec{\nabla} + \kappa$, the Hamiltonian is simplified to (also using $\bar{\psi}\gamma^0 = \psi^\dagger \gamma^0 \gamma^0 = \psi^\dagger)$
	\begin{center}
		\boxed{H = \int d^3\vec{x}\ i\psi^\dagger \dot{\psi}}
	\end{center}
	
	
	\section*{Solving the Dirac equation in vacuum}
	
	First, note that with the {\it ansatz} \[ \psi(x) = u(k) e^{-ik\cdot x},\quad k\cdot x = \vec{k}\cdot\vec{x} - \underbrace{k^0 ct}_{\omega_k t} \]
	
	\[ \omega_k = \pm c\sqrt{\vec{k}^2 + \kappa^2 }\]
	which is to say
	
	\begin{center} \boxed{ k^0 = \pm \sqrt{\vec{k}^2 + \kappa^2} }\end{center}
	
	
	The case $k^0 = + \sqrt{\vec{k}^2 + \kappa^2}$
	
	\[ (i\gamma^\mu\partial_\mu - \kappa) u(k) e^{-ikx} = 0\]
	
	Given that $\partial_\mu e^{-ikx} = -ik_\mu e^{-ikx}$
	
	\[ \Rightarrow (\gamma^\mu k_\mu - \kappa) u(k) \cancel{e^{-ikx}} = 0\]
	
	rearranging this
	\[ (k \cdot \gamma) u(k) = \kappa u(k)\]
	
	That is to say, $u(k)$ is an eigenvector of $k\cdot\gamma$ with eigenvalue $\kappa$
	
	\subsection*{Strategy I: Problem for $k=0$}
	i.e. $k^0 = \kappa$
	
	Apply a boost to the obtain the solution for general $\vec{k}$. So here $k\cdot \gamma^\mu = k^0\gamma^0$
	
	\begin{align*}
		\kappa \gamma^0 u(k) &= \kappa u(k)\\
		\gamma^0 u(k) &= u(k)
	\end{align*}
	In our choice of Dirac matrices
	
	\[ \gamma^\mu = \begin{pmatrix}
		0 & \sigma^\mu \\
		\bar{\sigma}^\mu & 0
	\end{pmatrix}, \sigma^\mu = \begin{pmatrix}
	\mathbb{1}_{1\times 1} \\
			\vec{\sigma}
\end{pmatrix}\ 
		\text{and}\ \bar{\sigma}^\mu = \begin{pmatrix}
		\mathbb{1}_{1\times1}\\ -\vec{\sigma}
\end{pmatrix}\]
	
	\[ \gamma^0 \begin{pmatrix}
		a \\ b
	\end{pmatrix} = \begin{pmatrix}
	b \\ a
\end{pmatrix}\]
	
	Eigenvectors with eigenvalue $+1$: $(k\cdot \gamma)U = kU$
	
\[\begin{pmatrix}
		a \\ a
	\end{pmatrix} \quad \forall a\in \mathbb{C}^2
\]
e.g. $a= \begin{pmatrix}
	1 \\ 0
\end{pmatrix}, a= \begin{pmatrix}
0\\ 1
\end{pmatrix}$ and $\begin{pmatrix}
	a \\ -a 
\end{pmatrix}$ have eigenvalue $-1,\ \forall a\in\mathbb{C}^2$\\


$\Rightarrow$ solution for $\vec{k} = 0$,

\[	  u(\vec{0}) = \sqrt{\kappa}\begin{pmatrix}
	\zeta \\ \zeta
\end{pmatrix}, \quad \zeta \in \mathbb{C}^2	 \]

Choose the normalization $\zeta^+\ \cdot \zeta = 1$

Boost that brings $\begin{pmatrix}
	k\\ \vec{\sigma}
\end{pmatrix}$ into the vector $\begin{pmatrix}
	k^0 = \sqrt{\vec{k}^2 + \kappa^2}\\ \
			\vec{k}
\end{pmatrix}$ is


		\[ \Lambda^\mu_\nu = \begin{pmatrix}
			\cosh \eta & 0 & 0 & \sinh \eta\\
			0 & 0 & 0 & 0 \\
			0 & 0 & 0 & 0\\
			\sinh \eta & 0 & 0 & \cosh \eta
		\end{pmatrix}\]

	$$k^\mu = \begin{pmatrix}
		k^0 \\ 0 \\ 0 \\ k^3
	\end{pmatrix} = (\Lambda^\mu_\nu)\begin{pmatrix}
	\kappa \\ 0 \\ 0 \\ 0
\end{pmatrix} = \begin{pmatrix}
		\kappa \cosh \eta\\ 0 \\ 0 \\ \kappa \sinh \eta
\end{pmatrix}$$

So this boost turns our rest frame wavevector with $k^0 = \kappa$ into a more general one

\[ \Rightarrow k^0 +  \k^3 = \kappa e^\eta\    \quad (\because\ \cosh x + \sinh x = e^{x})\]

and \[ k^0 - k^3 = \kappa e^{-\eta} \]

	So now e have to transform our spinor $u$ in a way that ... We know how it transforms, it's not with the matrix $\Lambda^\mu_\nu$, but with the spinor representations (using the matrices $\S^{\mu\nu}$).
	
	\subsection*{Transformation of the spinor}
	
		\[ \psi'(x') = \exp(-\frac{i}{2} 
		\omega_{\mu\nu} S^{\mu\nu})\psi(x)\]
	
	
	Here, we pick $\omega_{03} = -\omega^{30} = \eta$, all other $\omega_{\mu\nu}$ vanish.
	
	\[ \Rightarrow \exp \left(-\frac{i}{2} \omega_{\rho\sigma} S^{\rho \sigma}\right) = \exp(-i\eta\delta^{03}) = \exp\left(-\frac{\eta}{2}\begin{pmatrix}
		\sigma^3 & 0\\ 0 & -\sigma^3
	\end{pmatrix}\right)\]

which is $$ = -\frac{i}{2}\begin{pmatrix}
	\sigma^3 & 0 \\ 0 & - \sigma^3
\end{pmatrix}$$

Note that exponentiating such a complicated $4\times 4$ matrix here reduces to exponentiating a $2\times2$ matrix because in $\exp \lambda = \sum_k^\infty \frac{\lambda^n}{
n!}$ the different matrices don't talk to each other \footnote{
	The even powers give $\mathbb{1}$ and the odd powers give $\sigma^3$\\

\text{if} $A^2 = \lambda\mathbb{1}$, $$e^{A} = \left( \sum_{\text{even}} \lambda^{n} \right) \mathbb{1} + \left(\sum_{\text{odd}} \frac{\lambda^{(n-1)}}{n!}A \right)= \cosh \lambda \mathbb{1} + \frac{1}{\lambda}\sinh \lambda \cdot A$$

if $A = \sqrt{\lambda} \mathbb{1} e^{\sqrt{\lambda}}$...

	}
	
	\[ \text{the exponential becomes}\ = \begin{pmatrix} \frac{\mathbb{1}-\sigma^3}{2}e^{i\eta}  + \frac{\mathbb{1}+\sigma^3}{2}e^{-i\eta}  & 0 \\
		 0 &  \frac{\mathbb{1}-\sigma^3}{2}e^{i\eta} +  \frac{\mathbb{1}+\sigma^3}{2}e^{-i\eta} \end{pmatrix}\]
	
	
	So, 
	
	\[ \exp\left(-\frac{i}{2} \omega_{\rho\sigma} S^{\rho\sigma} \right) 
		= 	\frac{1}{\sqrt{k}}
	\begin{bmatrix}
		 \begin{pmatrix}
			\frac{\mathbb{1}-\sigma^3}{2}\sqrt{k^0 + k^3}  - \frac{\mathbb{1}+\sigma^3}{2}\sqrt{k^0 - k^3} 
		\end{pmatrix} & 0 \\
		
		0 & \begin{pmatrix}
			\sqrt{k^0 + k^3} \frac{\mathbb{1} - \sigma^3}{2} - \sqrt{k^0 - k^3}\left(\frac{1 + \sigma^3}{2}\right)
		\end{pmatrix}
	\end{bmatrix}
	\]
	
	What does the equation really say? These matrices (e.g. $\sigma^3$) are projectors. Which means if you square the matrix, you get the same matrix. 
	
	\begin{align*}
	\sigma^3 = \begin{pmatrix}
		1 & 0 \\ 0 & -1
	\end{pmatrix}, &\ \frac{1+\sigma^3}{2},\ \text{projects onto subspace with eigenvalue } +1\\
	&\ \frac{1-\sigma^3}{2},\ \text{projects onto subspace with eigenvalue } -1
	\end{align*}

		\textbf{General k}: $$u(k) = \sqrt{k} \begin{pmatrix}
			\sqrt{k^0 \mathbb{1} - \vec{k}\vec{\sigma}}\\
			\sqrt{k^0\mathbb{1} + \vec{k}\vec{\sigma}}
		\end{pmatrix} = \begin{pmatrix}
		\sqrt{k\cdot \sigma}\\ \sqrt{k\cdot \bar{\sigma}}
	\end{pmatrix}$$

\end{document}

