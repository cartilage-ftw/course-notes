\documentclass[11pt]{article}

\usepackage{geometry}
\usepackage{amsmath}
\usepackage{amsfonts}
%\usepackage[regular]{newcomputermodern}
\usepackage{unicode-math}
%\setmathfont[range=\mathbb]{TeX Gyre Termes Math}
%\usepackage{mathtools}
%\usepackage[T1]{fontenc}
\usepackage{cancel}
\usepackage{graphicx}
\usepackage{xcolor}
\usepackage{hyperref}
\hypersetup{colorlinks=true, urlcolor=blue, linkcolor=blue, citecolor=blue}
\urlstyle{same}

\geometry{margin=2.5cm, top=1cm}
\author{Aayush Arya}
\title{}
\date{January 23, 2023}

\newcommand{\lag}{\mathcal{L}}
\newcommand{\ham}{\mathcal{H}}

\begin{document}
	\maketitle
	
	\hrule
	\begin{center}
		Lecture Notes\\
		Advanced Quantum Mechanics
	\end{center}
	\hrule 
	
	\vspace{11pt}
	For constructing a field theory of fermions, we need to find a Lagrangian density $\lag$ that is Lorentz invariant, using some combination of $\psi(x)$ and its conjugate transpose.
	
	Transforming $\psi^\dagger(x)$ works in an analogous fashion to $\psi(x)$, by taking the conjugate transpose of the whole expression \[ \psi'(x') = \left(1 - \frac{i}{2}\omega_{\mu \nu}S^{\mu\nu}\right) \psi(x)\]
	and thereby getting
	\[ \psi'^\dagger (x') = \psi^\dagger(x)\left(1 + \frac{i}{2} \omega_{\mu\nu}S^{\mu\nu \dagger} \right)\]

	A couple of things to remember about the generators:
	\begin{align*}
		S^{\mu\nu}: \text{Hermitian for rotations}&\ (S^{ij};\ i,j > 0)\\
		 \text{and anti-Hermitian for boosts}&\ (S^{0j}\  \text{for}\ j > 0)\\
		 \text{also, the property}\ S^{\mu\nu\dagger} \rightarrow \gamma^0 S^{\mu\nu} \gamma^0 &\  \text{holds}
	\end{align*}
	This can be understood as follows: $\gamma^0$ is equivalent to the parity operator $P$.\\
	 \----\ Therefore $\gamma^0 S^{\mu\nu} \gamma^0 = P^{-1} S^{\mu\nu} P$ (since $\gamma^0$ is unitary, ($\gamma^0)^{-1} = \gamma^0$). This transformation will either cause a sign flip as we obtain $S^{\mu\nu\dagger}$ (for the anti-Hermitian boost matrices) or remain the same (for rotation matrices).\\
	
	We define $\bar{\psi} \equiv \psi^\dagger \gamma^0$, such that multiplying both sides by $\gamma^0$
	
	\[\psi'^\dagger(x')\gamma^0 = \psi^\dagger(x) \left(1 + \frac{i}{2}\omega_{\mu \nu}\gamma^0 S^{\mu\nu}\gamma^0 \right) \gamma^0\]
	since $(\gamma^0)^2 = 1$,
	\[ \bar{\psi}'(x') = \psi^\dagger (x) \left( \gamma^0 + \frac{i}{2}\omega_{\mu\nu}\gamma^0 S^{\mu\nu}\right)\]
	
	we can then pull out a $\gamma^0$ from the right, which will act on the $\psi^\dagger$
	
	\begin{center}
		\boxed{\bar{\psi}'(x') = \bar{\psi}(x) \left( \mathbb{1} + \frac{i}{2}\omega_{\mu\nu} S^{\mu\nu}\right)}
	\end{center}
	
	The product
	\begin{align*}
		\bar{\psi}'(x') \psi'(x') &= \bar{\psi}(x) \left(\mathbb{1} + \frac{i}{2}\omega_{\mu\nu} S^{\mu\nu} \right) \left( \mathbb{1} - \frac{i}{2}\omega_{\mu\nu}S^{\mu\nu} \right) \psi(x) \\
				& \sim \bar{\psi}(x)\psi(x),\ \text{if we ignore}\ \mathcal{O}(\omega_{\mu\nu}^2)\ \text{terms}\\
		\Rightarrow (\bar{\psi}\psi)'(x') &= (\bar{\psi}\psi) (x)
	\end{align*}

Therefore \underline{$\bar{\psi}\psi$ transforms like a scalar field}.

We may ask, how does $\bar{\psi}(x)\gamma^\mu \psi(x)$ transform?\\

As it turns out, the transformed $\bar{\psi}'(x')\gamma^\mu\psi'(x')$ takes up the form\footnote{I didn't understand this part. Rohan scribbled that the commutator for $S^{\mu\nu}$ and $\gamma^\mu$ pops up (figure it out).}

\[ \bar{\psi}(x') \gamma^\mu \psi(x') = \left(\delta^\mu_\sigma - \frac{i}{2}\omega_{\lambda\nu}(J^{\lambda\nu})^\mu_\sigma\right)\bar{\psi}(x)\gamma^\mu \psi (x) \]

same transformation is used for vector field ($\xleftarrow{?}$)\\

\noindent However, $\bar{\psi}(x)\gamma^\mu \partial_\mu \psi(x)$ will transform like a scalar field.\\
What went missing in $\bar{\psi}(x)\gamma^\mu\partial_{\mu}\psi(x)$? Let's remind ourselves
\begin{align*}
	\text{Covariant vectors:}\ x'_\mu =&\ (\Lambda^{-1})^\nu_\mu x_\nu\\
	\text{Contra:}\ x'^\mu =&\ \Lambda^\mu_\nu x^\nu\\
	x'_\mu x'^\mu =&\ x_\mu x^\mu, \quad \left[\because\ (\Lambda^{-1})^\nu_\mu (\Lambda)^\mu_\lambda = \delta^\nu_\lambda\right] 
\end{align*}
The way $\partial_{\mu}$ changes is
	 \[\frac{\partial}{\partial x'^\mu} = \left((\Lambda^{-1})^\nu_\mu \frac{\partial}{\partial x^\nu}\right)\]
\[ \hookrightarrow \text{this transformation was missing}\]
	
\subsection*{Constructing a Lagrangian}
For a scalar field we used $\phi, \phi^*$ and $\partial^\mu \phi^* \partial_\mu\phi$. For the spinor field, the following $\lag$ seems to recover the Dirac equation as its equation of motion.

	\[ \lag = i\bar{\psi}\gamma^\mu\partial_\mu\psi - k\bar{\psi}\psi \]
	
	where $$\bar{\psi}\gamma^\mu\partial_\mu\psi = \bar{\psi}\gamma^0 \frac{1}{c}\frac{d\psi}{dt} + \bar{\psi}\vec{\gamma}\cdot \nabla \psi$$
	
	and our $\vec{\gamma} = \begin{pmatrix}\gamma^1 \\ \gamma^2 \\ \gamma^3 \end{pmatrix}$

It's worth noting that since $\gamma^0$ is Hermitian (the time-like part), and
$\vec{\gamma}$ is anti-Hermitian (the spatial part), raising and lowering indices has the effect

\begin{align*}
	\gamma^0 = &\ \gamma_0\\
	\gamma_i =&\ -\gamma^i \\
	\gamma_\mu =&\ g_{\mu\nu} \gamma^\nu
\end{align*}

and so apparently, the metric tensor can take care of the signs. \\

Now, we derive the equation of motion for the Lagrangian we constructed from $\psi$

\[ \partial_\mu\left(\frac{\partial \lag}{ \partial(\partial_\mu\phi_l)}\right) = \frac{\partial\lag}{\partial \phi_l}, \quad l=1, .. N\]

We are treating $\psi$ and $\bar{\psi}$ as independent variables\\

Computing these quantities yields

\begin{center} \boxed{(i\gamma^\mu \partial_\mu - \kappa) \psi(x) = 0} \end{center} which is the celebrated Dirac equation.

There's an analogous equation for $\bar{\psi}$. If we take the conjugate transpose of the above equation, then multiply by $\gamma^0$ on the right, 

\[ (-i\partial_\mu\psi^\dagger\gamma^{\mu\dagger} - \psi^\dagger(x)\kappa)\gamma^0 = 0\]
insert the identity $\gamma^0\gamma^0 = 1$ in between $\partial_\mu\psi^\dagger$ and $\gamma^{\mu\dagger}$, and make use of $\gamma^0 \gamma^{\mu\dagger} \gamma^0 = \gamma^\mu$

\[ (i \underbrace{
	\partial_\mu\psi^\dagger\gamma^0}_{\partial_\mu \bar{\psi}} \underbrace{
		\gamma^0 \gamma^{\mu\dagger} \gamma^0 }_{\gamma^\mu} +
	 \kappa \psi^\dagger(x)\gamma^0) = 0\]
		
	after cleaning this up, we arrive at
	
\begin{center}
	\boxed{\bar{\psi}(x) (i\gamma^\mu\partial_\mu + \kappa) = 0 }
\end{center}

$\Rightarrow$ In QED, there would be extra interaction terms in the Lagrangian

\[ \lag = \bar{\psi} i \gamma^\mu (\partial_\mu - ieA_\mu)\psi - m\bar{\psi}\psi - \frac{1}{4}F_{\mu\nu}F^{\mu\nu}\]

\subsection*{Relevance of the mass  term in the Dirac equation}

Let's take
\[ (i \gamma^\mu \partial_\mu - \kappa)\psi(x) = 0 \quad \quad \times \left(i\gamma^\mu\partial_\mu + \kappa \right)\ \text{on the left}\]
such that
\[ (i\gamma^\mu \partial_\mu + \kappa)(i\gamma^\mu \partial_\mu - \kappa) \psi(x) = 0\]
and doing the multiplication gives
\[ (-\underline{\gamma^\nu\gamma^\mu} \partial_\nu \partial_\mu - \kappa^2)\psi(x) = 0 \]
the underlined $\gamma^\nu\gamma^\mu$, due to the anticommutator relation
\[ \frac{1}{2}(\gamma^\nu \gamma^\mu + \gamma^\mu\gamma^\nu) = \frac{1}{2} \{\gamma^\nu, \gamma^\mu\} = g^{\mu\nu}\mathbb{1}\]
gives
\begin{align*}
	\Rightarrow (-g^{\mu\nu} \partial_\nu \partial_\mu - \kappa^2) \psi(x) &= 0\\
	(\partial^\mu\partial_\mu - \kappa^2)\psi(x) &= 0\\
\end{align*}  
which is the Klein-Gordon equation.\\

Therefore, the Dirac equation implies the Klein-Gordon equation.


\end{document}

