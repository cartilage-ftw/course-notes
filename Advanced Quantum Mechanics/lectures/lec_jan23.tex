\documentclass[11pt]{article}

\usepackage{geometry}
\usepackage{amsmath}
\usepackage{amsfonts}
\usepackage[regular]{newcomputermodern}
%#\usepackage{unicode-math}
%\setmathfont[range=\mathbb]{TeX Gyre Termes Math}
%\usepackage{mathtools}
%\usepackage[T1]{fontenc}
\usepackage{cancel}
\usepackage{graphicx}
\usepackage{xcolor}

\geometry{margin=2.5cm, top=1cm}
\author{Aayush Arya}
\title{}
\date{January 23, 2023}

\newcommand{\lag}{\mathcal{L}}
\newcommand{\ham}{\mathcal{H}}

\begin{document}
	\maketitle
	
	\hrule
	\begin{center}
		Lecture Notes\\
		Advanced Quantum Mechanics
	\end{center}
	\hrule 
	
	\vspace{11pt}
	Transforming $\psi^\dagger(x)$ \[ \psi'(x') = \left(1 - \frac{i}{2}\omega_{\mu \nu}S^{\mu\nu}\right) \psi(x)\]
	
	\[ \psi'^\dagger (x') = \psi^\dagger(x)\left(1 + \frac{i}{2} \omega_{\mu\nu}S^{\mu\nu \dagger} \right)\]

	\begin{align*}
		S^{\mu\nu}: \text{Hermitian for rotations}&\ (S^{ij};\ i,j > 0)\\
		 \text{and anti-Hermitian for boosts}&\ (S^{0j}\  \text{for}\ j > 0)\\
		 \text{Property:}\ S^{\mu\nu\dagger} \rightarrow \gamma^0 S^{\mu\nu} \gamma^0 &\ \quad \text{(Why?)}
	\end{align*}
	
	So, \[\psi'^\dagger(x')\gamma^0 = \psi^\dagger(x) \left(1 + \frac{i}{2}\omega_{\mu \nu}\gamma^0 S^{\mu\nu}\gamma^0 \right) \gamma^0\]
	
	\[ \bar{\psi}'(x') = \psi^\dagger (x) \left( \gamma^0 + \frac{i}{2}\omega_{\mu\nu}\gamma^0 S^{\mu\nu}\right)\]
	
	%\section*{The Lorentz Group}
	
	
	
	\[ \bar{\psi}'(x') = \bar{\psi}(x) \left( \mathbb{1} + \frac{i}{2}\omega_{\mu\nu} S^{\mu\nu}\right) \]
	
	The product
	\begin{align*}
		\bar{\psi}'(x') \psi'(x') &= \bar{\psi}(x) \left(\mathbb{1} + \frac{i}{2}\omega_{\mu\nu} S^{\mu\nu} \right) \left( \mathbb{1} - \frac{i}{2}\omega_{\mu\nu}S^{\mu\nu} \right)\\
				& \sim \bar{\psi}(x)\psi(x)\\
		(\bar{\psi}\psi)'(x') &= (\bar{\psi}\psi) (x)
	\end{align*}

Therefore $\bar{\psi}\psi$ transforms like a scalar field.

Transformation of $\bar{\psi}(x)\gamma^\mu \psi(x)$ under Lorentz transform\\

Transformed: $\bar{\psi}'(x')\gamma^\mu\psi'(x')$ use the transformations if ... (I didn't understand this part)

The commutator for $S^{\mu\nu}$ and $\gamma^\mu$ pops up (figure it out)

Takes up a form,
\[ \bar{\psi}(x') \gamma^\mu \psi(x') = \left(\delta^\mu_\sigma - \frac{i}{2}\omega_{\lambda\nu}(J^{\lambda\nu})^\mu_\sigma\right)\bar{\psi}(x)\gamma^\mu \psi (x) \]

same transformation is used for vector field \\

$\bar{\psi}(x)\gamma^\mu \partial_\mu \psi(x)$ will transform like a scalar field.\\

\begin{align*}
	\text{Covariant vectors:}\ x'_\mu =&\ (\Lambda^{-1})^\nu_\mu x_\nu\\
	\text{Contra:}\ x'^\mu =&\ \Lambda^\mu_\nu x^\nu\\
	x'_\mu x'^\mu =&\ x_\mu x^\mu, \quad \left[\because\ (\Lambda^{-1})^\nu_\mu (\Lambda)^\mu_\lambda = \delta^\nu_\lambda\right] 
\end{align*}

	 \[\frac{\partial}{\partial x'^\mu} = \left((\Lambda^{-1})^\nu_\mu \frac{\partial}{\partial x^\nu}\right)\]
\[ \hookrightarrow \text{this transformation was missing}\]
	
\subsection*{Constructing a Lagrangian}
For a scalar field we used $\phi \phi^*$ (i.e. $\partial^\mu \phi^* \partial_\mu\phi$). For the fields under a Lorentz transformation 

\begin{align*}
	\lag =& i\bar{\psi}\gamma^\mu\partial_\mu\psi - k\bar{\psi}\psi\\
	& \hookrightarrow \bar{\psi}\gamma^0 \frac{1}{c}\frac{d\psi}{dt} + \bar{\psi}\vec{\gamma}\cdot \nabla \psi\\
	\text{with}\ & \vec{\gamma} = \begin{pmatrix}\gamma^1 \\ \gamma^2 \\ \gamma^3 \end{pmatrix} \\
\end{align*}

$\gamma^0 \rightarrow$ Hermitian (time-like part)\\
$\vec{\gamma} \rightarrow$ anti-Hermitian (spatial part)


\begin{align*}
	\gamma^0 = &\ \gamma_0\\ \gamma_\mu =&\ -\gamma^\mu \\
	\gamma_\mu =&\ g_{\mu\nu} \gamma^\nu
\end{align*}

the metric tensor transforms like the Dirac matrices\\

Equation of motion for the constructed lagrangian of $\psi$

\[ \partial_\mu\left(\frac{\partial \lag}{ \partial(\partial_\mu\phi_l)}\right) = \frac{\partial\lag}{\partial \phi_l}, \quad l=1, .. N\]

We are treating $\psi$ and $\bar{\psi}$ as independent variables

So, for the Lagrangian,

\[ (i\gamma^\mu \partial_\mu - \kappa) \psi(x) = 0 \rightarrow \text{Dirac equation}\]

The equation for $\bar{\psi}$ turns out to be 
\[ \bar{\psi}(x) (i\gamma^\mu\partial_\mu + \kappa) = 0\]
$$\hookrightarrow \text{Derive this}$$
(Just multiply Dirac equation by $\gamma^0$ and pull out a $\gamma^0$ after taking conjugate).

So for QED, there would be extra interaction terms in the Lagrangian

\[ \lag = \bar{\psi} i \gamma^\mu (\partial_\mu - ieA_\mu)\psi - m\bar{\psi}\psi - \frac{1}{4}F_{\mu\nu}F^{\mu\nu}\]

Relevance of $\kappa$ in the Dirac equation

\[ (i \gamma^\mu \partial_\mu - \kappa)\psi(x) = 0 \quad \quad \times \left(i\gamma^\mu\partial_\mu + \kappa \right)\ \text{on the left}\]

\[ (i\gamma^\mu \partial_\mu + \kappa)(i\gamma^\mu \partial_\mu - \kappa) \psi(x) = 0\]
\[ (-\underbrace{\gamma^\nu\gamma^\mu}_{\text{becomes}}\partial_\nu \partial_\mu - \kappa^2)\psi(x) = 0 \]
\[ \hookrightarrow \frac{1}{2}(\gamma^\nu \gamma^\mu + \gamma^\mu\gamma^\nu) = \frac{1}{2} \{\gamma^\nu, \gamma^\mu\} = g^{\mu\nu}\mathbb{1}\]

\begin{align*}
	\Rightarrow (-g^{\mu\nu} \partial_\nu \partial_\mu - \kappa^2) \psi(x) &= 0\\
	(\partial^\mu\partial_\mu - \kappa^2)\psi(x) &= 0\\
	\hookrightarrow \text{ Klein-Gordon equation}
\end{align*}  

Therefore, the Dirac equation implies the Klein-Gordon equation.


\end{document}

