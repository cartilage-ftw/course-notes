% compile with XeTeX
\documentclass[11pt]{article}

\usepackage{geometry}
\usepackage{amsmath}
\usepackage{amsfonts}
%\usepackage[regular]{newcomputermodern}
\usepackage{unicode-math}
%\setmathfont[range=\mathbb]{TeX Gyre Termes Math}
%\usepackage{mathtools}
%\usepackage[T1]{fontenc}
\usepackage{cancel}
\usepackage{graphicx}
\usepackage{xcolor}

\usepackage{fancyhdr}

\usepackage{slashed} % for Dirac operators

\usepackage{hyperref}
\hypersetup{colorlinks=true, linkcolor=blue, citecolor=blue, urlcolor=blue}

\geometry{margin=2.5cm, top=2.5cm}
\author{}
\title{}
\date{}

\newcommand{\lag}{\mathcal{L}}
\newcommand{\ham}{\mathcal{H}}

\begin{document}

	\pagestyle{fancy}
	\maketitle
	
	\vspace{-8em}
	
 	
	%\hrule
	\noindent
	%\hspace{-1.2em}
	\fbox{
		\parbox{\textwidth}{
	\begin{center}
	\vspace{-8pt}
	\footnotesize{Advanced Quantum Mechanics}\hfill \footnotesize{WiSe 2022/23}\\
	%\vspace{4pt}
		\centering \normalsize Aayush Arya \\
		\centering \small \today\\
		\footnotesize{Theoretical Physics 5} \hfill \footnotesize{Instructor: Harvey Meyer}
	\end{center}
	\vspace{-8pt}
		}
	}
	%\hrule 
	
	\vspace{11pt}
	
	
	\section*{Recap}
	
	\[ (\gamma^\mu k_\mu - \kappa) u(k) =  \begin{pmatrix} 0 & k\cdot \sigma \\ k\cdot \bar{\sigma} & 0 \end{pmatrix} \begin{pmatrix}
		\sqrt{k\cdot \sigma} \\ \sqrt{k \cdot \bar{\sigma}}
	\end{pmatrix} = \begin{pmatrix}
		(k \cdot \sigma ) \sqrt{k \cdot \bar{\sigma}} \\ (k\cdot \bar{\sigma})\sqrt{k\cdot \sigma}
\end{pmatrix} \]
	
	
	
	\[ (k\cdot \sigma) k\cdot \bar{\sigma}  = k^2 \mathbb{1}_{2\times2}\]
	
	\textbf{Indeed}: $k_\mu \sigma^\mu k_\nu \bar{\sigma}^\nu = k_\mu k_\nu \frac{1}{2}\{\sigma^\mu, \bar{{\sigma}}^\nu \}$
	
	$$ = k_\mu k_\nu \frac{1}{2} (\sigma^\mu \bar{{\sigma}}^\mu + \sigma^\nu \bar{\sigma}^\nu )$$
	
	%\[ = \left\{ \mathbb{1}, \mu = \nu = 0; \quad -\delta^{ij}\mathbb{1}, \mu=i,\ v=j, 0\right\}\]
	
	\[ = \begin{cases}
		\mathbb{1} & \mu = \nu = 0\\
		-\delta^{ij}  & \mu = i, \nu = j\\
		0 & \mu = 0, v=j\\
		0 & \mu=i, \nu=0
	\end{cases}\]
	
	
	\[ = g^{\mu\nu} \mathbb{1}_{2\times 2}\]
	
	
	Another ansatz: \[ \psi(x) = v(k) e^{+ikx}\]
			for the antiparticle case, but also with $k^0 = + \sqrt{k^2 + \kappa^2}$.
			
			That is to say, $k^0$ is still kept positive (positive energy, negative frequency).
	
				Putting this back into the Dirac equation
				
				\[ (-\gamma^\mu k_\mu - \kappa) v(k) = 0 \]


				\[ \implies (\gamma^\mu k_\mu + \kappa)\ v(k) = 0 \]


			At $\vec{k} = 0$: \[ (\gamma^0 + 1)\ v(\vec{k} = 0) = 0 \]


		the eigenvector of $\gamma^0$ with digenvalue $_1$ \[ v(\vec{k} = 0) = \begin{pmatrix}
			\eta \\ -\eta
		\end{pmatrix}, \eta \in \mathbb{C}^2\]
	
	
		Apply boost to obtain $v(k)$ for a general $k$ (not at rest).
		
		\[ v(k) = \begin{pmatrix}
			\sqrt{k\cdot \sigma}\ \eta\\
			-\sqrt{k\cdot \bar{{\sigma}}\ \eta}
		\end{pmatrix}\]


		\subsection*{The full set of solutions of Dirac equation}
		
		 \[ \psi(x) = \begin{pmatrix}
		 	\sqrt{k\cdot \sigma}\\
		 	\sqrt{k\cdot \bar{\sigma}}
		 \end{pmatrix} e^{-ik\cdot x}\]
		
and \[ 
			\psi(x) = \begin{pmatrix}
				\sqrt{k\cdot \sigma}\ \eta \\
				- \sqrt{k\cdot \sigma}\ \eta 
			\end{pmatrix} e^{ik\cdot x}
		\]
		
		
		Basis of solutions  $\xi = \begin{pmatrix}
			1 \\ 0 
		\end{pmatrix}.\ \xi = \begin{pmatrix}
		0 \\ 1
	\end{pmatrix}$

and $ \eta = \begin{pmatrix}
	1 \\ 0
\end{pmatrix}, \eta = \begin{pmatrix}
0 \\ 1
\end{pmatrix}$
	\\

	
	
	Orthogonality and completenesss of the solutions 
	
	\[ \bar{u}^r (k) u^s (k) = 2\kappa \xi^{r\dagger} \cdot \xi^s = 2\kappa \delta^{rs} ; \quad \bar{u} \equiv u^\dagger\cdot \gamma^0 \]	
	

			The same also holds for $\bar{v}^r(k) v^s(k) = -2\kappa \delta^{rs}	$
		
		Additionally \[ \bar{u}^r (k) v^s(k) = 0,\ \forall r,s \in \{1, 2\} \]
		and same goes the other way
		
		
		
			Another set of relations
			
			\[ u^{r\dagger} (k)\ u^s (k) = 2\frac{\omega_k}{c} \delta^{rs} = 2k^0 \delta^{rs}\]
			 
			 
			 \[ v^{r\dagger}(k)\ v^s(k) = 2k^0 \delta^{rs} \]
			 
			 
			 \textbf{Note}: $u^{r\dagger} v^s(k) \neq 0$ 
			 but $u^r(\vec{k})^\dagger\ v^s(\vec{k}) = 0$\\
			 
		
			``Spin sum rules": $\gamma^\mu k_\mu \equiv \slashed{k}$ (Feynman notation)	
			
			\[ \sum_{s=1,2} u^s(k) \bar{u}^s (k)  = \slashed{k} + \kappa \]
			 note that this $\slashed{k} + \kappa$ is a projector (i.e., $P^2 = P$)
			 
			 \[ \slashed{p} \cdot \slashed{k} = (p\cdot k) \mathbb{1}_{4\times 4}\]
			 
			 \[ \left( \frac{\slashed{k} + \kappa}{2\kappa} \right) = \frac{1}{4\kappa^2} (\underbrace{k^2}_{\kappa^2} + 2\kappa k + \kappa^2)  = \frac{1}{2\kappa} (k + \slashed{k})\]
			   
			 \begin{center}
			 	\boxed{\sum_{r=1,2} v^r(k)\ \bar{v}^r(k) = \slashed{k} - \kappa }
			 \end{center}
			 
			 
			This is analogous to \[ \sum_{\lambda=1,2} e^i_\lambda (k) e^j_\lambda (k) = \delta^{ij} \frac{k^i \cdot \k^j}{\vec{k}^2}\]
			 
			 
			 
			 \subsection*{Behavior of the solutions for $k^3 \to $infty}
			 
			 We've already discussed the special case when our $\vec{k} = 0$. 
			 
			 For $k^3 \to \infty$, we can forget $k^1 = k^2 = 0$
			 
			 \[ u(k)^{s=1} = \begin{pmatrix}
			 	\sqrt{k^0 - k^3} \begin{pmatrix}
			 		1 \\ 0
			 	\end{pmatrix}\\
		 	
		 		\sqrt{k^0  + k^3} \begin{pmatrix}
		 			1 \\ 0
		 		\end{pmatrix}
			 \end{pmatrix}  \xrightarrow{k^3 \to \infty}\ = \sqrt{2|k^3| } \begin{pmatrix}
			 	0 \\ 0 \\ 1 \\ 0
			 \end{pmatrix}\]
			 
			 
			 
			 \[ u^{s=2} (k) = \begin{pmatrix}
			 	\sqrt{k^0 + k^3} \begin{pmatrix}
			 		0 \\ 1
			 	\end{pmatrix}\\
		 	\sqrt{k^0 - k^3} \begin{pmatrix}
		 		0 \\ 1
		 	\end{pmatrix}
			 \end{pmatrix}  \xrightarrow{k^3\to + \infty}\  = \sqrt{2|k^3|} \begin{pmatrix}
			 	0 \\ 1 \\ 0 \\ 0
		 \end{pmatrix}}\]
			 
			 
			 This also corresponds to the limit $\kappa \to 0$
			 
			 In this limit, only two components survive.\\
			 
			 
			 $\Rightarrow$ Why did Dirac choose a 4 dimensional representation of the Lorentz group? This looks like a waste of 2 degrees of freedom.\\
			 
			 It's because we wanted to describe a particle of a finite mass (e.g. electron). If we wanted to describe a massless fermion of spin$-1/2$ then
			 we could make do with just the upper two components in the block-diagonal $$S^{\mu\nu} = \begin{pmatrix}
			 	* & 0 \\ 0 & *
			 \end{pmatrix}$$
			 and that would be a perfectly fine representation of the Lorentz group.\\
	
	
		The mass term in the $\lag$, that is $-\kappa\bar{\psi}\psi$ couples the upper two and the lower two components.
			Otherwise we wouldn't need that many degrees of freedom.\\
	
	
			\fbox{ \parbox{\textwidth}{$\Rightarrow$  To describe waves (read: particles) with a dispersion relation $\kappa \neq 0$, it is necessary to use four-component spinors.}}\\
			
			
				For $\kappa = 0$, a two-component spinor suffices.\\
				
			
			
			\begin{align*}
			\textbf{Define}:	\gamma^5 &:= i\gamma^0 \gamma^1 \gamma^2 \gamma^3\\
						& = i \epsilon_{\mu\nu\rho\sigma} \gamma^\mu \gamma^\nu \gamma^\rho \gamma^\sigma \\
						& = \text{Lorentz scalar}
			\end{align*}
			 
			 Actually, this is a pseudo-scalar. Under $T$ and $P$ reversal, this picks up a $-$ sign. But for the purpose of a \textit{proper orthochronous} Lorentz group, this is sufficient.
			 
			 \[ \gamma^5 = \begin{pmatrix}
			 	-\mathbb{1}_{2\times 2} & 0 \\
			 	0 & \mathbb{1}_{2\times 2}
			 \end{pmatrix}\]
			 
			 
			 \[ \frac{\mathbb{1} + \gamma^5}{2} = \begin{pmatrix}
			 	0 & 0 \\ 0 & \mathbb{1}
			 \end{pmatrix} : \text{Projector onto the 2 lower components}\]
		 
		 	\[ \frac{\mathbb{1} - \gamma^5}{2}  = \begin{pmatrix}
		 		\mathbb{1} & 0 \\ 0 & 0
		 	\end{pmatrix} : \text{Projector onto the upper two components}\]
			 
			 
	\end{document}